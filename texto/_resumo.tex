\clearpage
\thispagestyle{empty}

\noindent{\large\bf\dadoTitulo}
\noindent{\large\dadoSubTitulo}

\normalsize
\begin{center}	
	\vspace*{0.5cm}
	\textbf{RESUMO}
\end{center}


A utilização de abordagens de teste no desenvolvimento de software com a finalidade de garantir qualidade e a segurança, vem crescendo exponencialmente entre os modelos de processos nas últimas décadas. Linha de Produto de Software é um modelo de processo onde um grupo de sistemas possuem funcionalidades similares, pois faz reutilização de componentes de software, obtendo maior produtividade, reduzindo tempo e custo. Já as características que diferenciam um produto dos demais em torno deste núcleo, damos o nome de variabilidade, e é ai um dos pontos que devemos levar em consideração quando abordamos teste em LPS, pois se torna um grande desafio realizar a cobertura de todas as funções variáveis de um grupo de produtos, além de que teste exaustivo é inviável. Porém, se faz importante a análise e o gerenciamento da variabilidade como um todo, pois a variabilidade representa diferentes tipos de variações sob diferentes níveis com diferentes tipos de dependências. Sobre Variabilidade, hoje, já existe algumas abordagens de gerenciamento, uma delas é o \textit{Stereotyp-based Management of Variability (SMarty)} ela guia o usuário na identificação e representação de variabilidades, formando um conjunto de estereótipos e meta-atributos para representar variabilidades em modelos UML de uma LPS.  A vantagem na utilização do TBM em LPS, seria pelo fato de quanto antes o teste ser realizado, maiores são a chances de garantia de qualidade, pensando nisso o objetivo geral deste trabalho visa a busca por evidências para a criação de uma abordagem utilizando teste baseado em modelo (TBM) em linhas de produtos de software (LPS) que também contenha gerenciamento de variabilidade utilizando a abordagem SMarty e que possam ser empregadas a nível de sistema a partir da engenharia de domínio. O Objetivo seria a conversão do modelo, que pode ser um caso de uso em um diagrama de sequência, a abordagem SMarty para identificação e classificação das variáveis, posteriormente uma possível conversão em Máquina de estado finito MEF e utilizar TBM para a geração dos casos de teste a partir do modelo SMarty, assim dando origem ao perfil \textit{SMartyTesting}. Com ela esperamos resultados positivos no que tange a qualidade em nível de sistema para todos os produtos derivados em uma LPS e como contribuição a utilização de TBM em SMarty.
 \\
 
\noindent 
\textbf{Palavras-chave:} Linha de produto de \textit{Software}. \textit{SMarty}. Teste Baseado em Modelo. Variabilidade. Qualidade de software.

\pagebreak
