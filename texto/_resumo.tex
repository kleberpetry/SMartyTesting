\clearpage
\thispagestyle{empty}

\noindent{\large\bf\dadoTitulo}
\noindent{\large\dadoSubTitulo}

\normalsize
\begin{center}	
	\vspace*{0.5cm}
	\textbf{RESUMO}
\end{center}
A utilização de reuso de código e de abordagens de teste no desenvolvimento de software, com a finalidade de garantir e aumentar a produtividade e a qualidade, vem crescendo exponencialmente entre os modelos de processos nas últimas décadas. Linha de Produto de Software (LPS) é um modelo de processo em que o reuso não oportunístico é o cerne do seu desenvolvimento. Levando em consideração a variabilidade inerente aos produtos derivados de uma LPS, uma forma efetiva de garantir a qualidade de tais produtos é a utilização de técnicas de teste. Para gerenciar variabilidades de uma LPS existem diversas abordagens, em especial as baseadas em \textit{Unified Modeling Language} (UML). A abordagem \textit{Stereotype-based Management of Variability (SMarty)} permite realizar tal gerenciamento. \textit{SMarty} guia o usuário na identificação e representação de variabilidades em modelos UML, por meio de estereótipos e meta-atributos. \textit{SMarty} oferece atualmente uma técnica de verificação de seus modelos na forma de inspeção baseada em \textit{checklists}. Porém, \textit{SMarty} não fornece uma forma de validação usando, por exemplo, Teste Baseado em Modelos (TBM). Com base nesse cenário e motivação, a busca por uma abordagem de geração de sequências de teste se faz necessária para a validação dos produtos instanciados. Assim, este trabalho teve como objetivo especificar uma abordagem para auxiliar na geração de sequências de teste, a partir de diagramas de sequência modelados com base em casos de uso e seus fluxos básicos e alternativos. Para avaliar tal abordagem foi realizado um estudo comparativo com outra abordagem existente na literatura, considerando quatro critérios de comparação: complexidade ciclomática, diferenciação das sequências, quantidade de sequências geradas e nível de esforço despendido na utilização da abordagem. Os resultados apontam viabilidade para utilização do modelo de abordagem proposta e, as contribuições são voltadas para a automatização dos processos, diminuição das etapas de tais processos e, suporte à programação concorrente para a ferramenta SPLiT-MBt.
\\

\noindent 
\textbf{Palavras-chave:} Linha de Produto de \textit{Software}. \textit{SMarty}. teste baseado em modelo. variabilidade. qualidade de \textit{software}.

\pagebreak
