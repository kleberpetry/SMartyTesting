\chapter{Introdução}
\label{sec:introducao}
\pagestyle{plain}

O mercado se torna cada vez mais competitivo, indicadores de investimento contabilizam e determinam a rentabilidade ou apresentam resultados desastrosos. Baseado nesse cenário, agilidade no desenvolvimento e qualidade de um produto se tornam um fator decisivo. Reutilização de código, diminuição de etapas no processo final são meios que podem ser utilizados para se aumentar a produtividade e rentabilidade, mas até que ponto isso é levado em prática. 
\section{Contextualização}
Diversas abordagens têm sido desenvolvidas com propósito de aumentar a reusabilidade de software e, consequentemente, o retorno de investimento (ROI) \cite{delamaro2017introduccao}. Entre essas abordagens, está Linha de Produto de Software (LPS). LPS é uma alternativa de utilização como processo de desenvolvimento baseado em reuso de software, para proporcionar maior produtividade, redução de custo, tempo, risco e proporcionar maior qualidade ao produto.

Uma LPS é um conjunto de sistemas de software que compartilham características (\textit{feature}) comuns e gerenciáveis, que satisfazem as necessidades de um segmento particular ou de uma missão \cite{clements2002software}. Esse conjunto de sistemas é denominado também, família de produtos.

Uma LPS possui artefato que podem ser reutilizados, assim, tendo em vista esse novo desafio de gerenciamento durante o desenvolvimento de software nota-se que em \textcolor{red}{publicados ***verificar fontes das publicações e adicionar aqui****} onde, os artefatos produzidos nos processos tradicionais de desenvolvimento de software já não são tão eficientes para o contexto de LPS, pois em sua maior parte não possui suporte a variabilidade. Assim, visando contornar este problema, \textcolor{red}{algumas abordagens foram propostas***citar as abordagens***} para o gerenciamento de variabilidade em LPS.

Uma das abordagens é a \textit{Stereotype-based Management of Variability} (\textit{SMarty}) \cite{junior2010systematic} , \textcolor{red}{que é composta de um perfil UML***exemplificar melhor essa parte***}, o \textit{SMartyProfile}, e do processo denominado \textit{SMartyProcess}. \textit{SMarty} tem como objetivo permitir que as variabilidades de uma LPS possam ser gerenciadas de forma clara e explícita em modelos UML, e guia o usuário por meio do \textit{SMartyProcess} na identificação e representação de variabilidades em tais modelos ou artefatos, que pode ser de caso de uso, classe, sequência, atividade e componentes de uma LPS.

\textit{SMarty} possui também um perfil de inspeção de software chamado \textit{SMartyCheck}, que visa a remoção de erros primários ou situações ligadas a requisitos. Embora é feito a verificação se faz importante a validação onde possa garantir uma maior qualidade dos modelos, para isso se faz pertinente testá-los.

\section{Motivação}
Desta forma, teste de software é uma abordagem essencial para a contribuição da geração de qualidade em um produto, o grande desafio é realizar isso em uma LPS, mesmo que ela seja uma abordagem que proporciona um padrão de processo o que permite gerar qualidade. Esse desafio se deve a grande quantidade de derivações de produtos que uma LPS pode permitir, também devemos levar em consideração que teste exaustivo é inviável \cite{do2014strategies}.

Outro fator que devemos levar em consideração quando se trata de teste e qualidade, é o inicio do ciclo de teste, pois, quanto mais cedo o ciclo se inicia, grandes são as probabilidades de uma maior cobertura de erros. Onde um problema descoberto muito tarde pode gerar um prejuízo muito maior do que se ele fosse detectado no inicio.

Assim, visualizando tais situações citadas, teste baseado em modelo (TBM) pode ser usado em conjunto com LPS, visando a qualidade do produto de software e a garantia de uma estrutura com maior reuso \textcolor{red}{como garantir esse reuso}. Nesse caso, surge oportunidade de investigar a possibilidade de criação de uma técnica de TBM em associação com a abordagem \textit{SMarty}, em que o teste de software em nível de modelagem onde possa ser reaproveitado em tempo de aplicação se apresente como algo interessante para garantir um certo nível de cobertura do núcleo de artefatos da LPS. Visualizando a utilização de LPS sob o conceito de engenharia de domínio e considerando a abordagem \textit{SMarty}, fica evidente a possibilidade de utilização de TBM em nível de funcionalidade, sob o domínio de modelagem e posteriormente o de aplicação onde poderia se permitir uma melhor visão sobre as variabilidades do produto de uma LPS.

Teste em LPS é sempre um desafio, devido ao fator variabilidade \textcolor{red}{explicar melhor porque o teste é um desafio quando se fala em variabilidade} \cite{chen2009variability,engstrom2011software,chen2011systematic,do2014strategies}. Neste caso, pensando em TBM, o teste em LPS é iniciando mais cedo, em tempo de modelagem, auxiliando na conversão do modelo em artefatos que possam ser reutilizados mais tarde, permitindo a identificação de variabilidade preservando suas características para possíveis soluções em tempo de aplicação. \cite{lamancha2010model,reales2011model,lamancha2009automated}. Por esse motivo, ele se candidata a ser uma técnica promissora para utilização em conjunto com \textit{SMarty}.

\section{Objetivo}
O objetivo deste trabalho é, portanto, especificar uma abordagem de TBM para modelos LPS modelados de acordo com a abordagem \textit{SMarty}. Para tanto, deve-se considerar a variabilidade explícita em tais modelos, especialmente os modelos comportamentais como diagramas de sequência \textit{State Machine}. Espera-se, assim, gerar sequências de teste e derivar cenários de teste para a criação de casos de teste a nível de sistema ou funcionalidade. Sendo assim, espera-se responder à seguinte questão de pesquisa: \textbf{É possível utilizar técnicas de TBM para se testar modelos \textit{SMarty} considerando variabilidade?}

\section{Método de Desenvolvimento}
Analisando por essa perspectiva, \textcolor{red}{acredita-se} que exista a necessidade de se realizar uma pesquisa aprofundada para se obter o estado da arte sobre TBM aplicado em LPS com foco em variabilidade, verificar também a automatização do processo de geração, níveis de cobertura e rastreabilidade.

\subsection{Porque Diagramas de Sequência?}
\label{sec:porque_sequencia}
Um terceiro ponto é a utilização de diagramas de sequência como ponto de partida, \citealp{MarcolinoICEIS2014} estruturou a ampliação do perfil \textit{SMarty} para suporte a diagramas de sequência, um modelo que possui uma maior quantidade de informações comparado com um diagrama de atividade por exemplo, com ele é possível a representação de laços de repetições assim como condicionais, assim quando convertido para um outro modelo de ele poderá manter as propriedade de estados, esperando assim gerar uma sequencia de teste maior e mais abrangente comparando com outros trabalhos que utilizem artefatos mais alto nível para comparativo com o trabalho de \citealp{costa2016split}.

\section{Organização do Texto}
O trabalho está organizado como segue. No capítulo dois é apresentada a fundamentação da área abordada, linha de produto de software e \textit{SMarty}, Teste Baseado em Modelo (TBM) e a abordagem SPLit-MbT que será base para este trabalho. O Capítulo três descreve a proposta assim como as ferramentas e meios para a aplicação e utilização de \textit{SMartyTesting}. O Capítulo quatro apresenta a avaliação realizada com a abordagem e o capítulo cinco finaliza com as conclusões. No apêndice são apresentados um mapeamento sistemático da literatura sobre TBM em LPS e um anexo sobre \textit{SMarty}.