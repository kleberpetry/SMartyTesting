\clearpage
\thispagestyle{empty}

\noindent{\large\bf\dadoTituloAbs}
%\noindent{\large\dadoSubTituloAbs}

\normalsize
\begin{center}	
	\vspace*{0.5cm}
	\textbf{\textit{ABSTRACT}}
\end{center}
The use of test approaches in software development to ensure quality and safety has been growing exponentially among process models in recent decades. Software Product Line is a process model where a group of systems have similar functionalities, since it makes reuse of software components, obtaining greater productivity, reducing time and cost. Since the characteristics that differentiate a product from the others around this nucleus, we call it variability, and this is one of the points that we must take into account when approaching test in LPS, since it becomes a great challenge to cover all the functions variables of a product group, in addition to which exhaustive testing is not feasible. However, it is important to analyze and manage the variability as a whole, since the variability represents different types of variations under different levels with different types of dependencies. On Variability, today, there are already some management approaches, one of which is the Stereotyp-based Management of Variability (SMarty), which guides the user in the identification and representation of variabilities, forming a set of stereotypes and meta-attributes for represent variabilities in UML models of an LPS. TBM is a form of software testing where the test cases are derived from a model that describes aspects (usually functional ) of the system being tested. The advantage of using TBM in LPS would be because the sooner the test is performed, the greater the chances of quality assurance, considering the general objective of this work is the search for evidence for the creation of a technique using test based (TBM) in software product lines (LPS) that also contain variability management using the SMarty approach and that can be employed at the system level from domain engineering. The objective would be the conversion of the model, which can be a use case in a sequence diagram, the SMarty approach for identification and classification of variables, later a possible conversion in MEF finite state machine and use TBM to generate the cases of test from the SMarty model, thus giving rise to the \ textit {SMartyTesting} profile. With it we expect significant results on the system-level quality for all products derived in a LPS and as contribution to the use of TBM in SMarty.
\\

\noindent \textbf{\textit{Keywords:}} Sfotware Product Line. SMarty. Model-Based Testing. Variability. Software Quality.

\pagebreak

 
