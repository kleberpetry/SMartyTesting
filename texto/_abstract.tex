\clearpage
\thispagestyle{empty}

\noindent{\large\bf\dadoTituloAbs}
%\noindent{\large\dadoSubTituloAbs}

\normalsize
\begin{center}	
	\vspace*{0.5cm}
	\textbf{\textit{ABSTRACT}}
\end{center}
The use of code reuse and testing approaches in software development to ensure and increase productivity and quality has grown exponentially among process models in recent decades. Software Product Line (SPL) is a process model in which non-opportunistic reuse is the core of its development. Given the inherent variability in products derived from an SPL, an effective way to ensure the quality of such products is to use testing techniques. To manage variability of an SPL there are several approaches, especially those based on UML. The Stereotype-based Management of Variability (SMarty) approach enables such management. SMarty guides the user in identifying and representing variability in UML models through stereotypes and meta-attributes. SMarty currently offers a verification technique for its models in the form of checklist-based inspection. However, SMarty does not provide a form of validation using, for example, Model Based Testing (MBT). Based on this scenario and motivation, the search for a test sequence generation approach is necessary to validate the instantiated products. Thus, this paper aims to specify an approach that assists in the generation of test sequences from sequence diagrams modeled based on use cases and their basic and alternative flows. To evaluate such an approach, a comparative study was performed with another approach in the literature considering four comparison criteria: cyclomatic complexity, sequence differentiation, number of sequences generated and level of effort spent in using the approach. The results indicate the feasibility of using this approach model and the contributions are directed to the automation of processes, reduction of the steps of such processes, concurrent programming support for the SPLiT-MBt tool.
\\

\noindent \textbf{\textit{Keywords:}} Software Product Line. SMarty. Model-Based Testing. Variability. Software Quality.

\pagebreak

 
