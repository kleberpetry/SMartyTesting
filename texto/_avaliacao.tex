\chapter{Avaliação Empírica de SMartyTesting}
\label{avaliacao}
\pagestyle{plain}

\section{Considerações Iniciais}
\section{Objetivo}
\section{Planejamento}
\begin{itemize}
	\item Estudo Qualitativo - Utilização do modelo proposto em uma LPS real com foco na criação de produtos mobile para ensino de programação, uma vez a LPS constituída, os produtos modelados em engenharia de domínio previamente modelados notados em \textit{SMarty} seria submetidos a \textit{SMartyTesting}, podendo até ser comparada com outras abordagens como SPLiT- MBT.
	\item Experimento controlado - Avaliar o modelo proposto com base em experimentos controlados, a partir de modelos \textit{SMarty} utilizando-se de diagrama de caso de uso e sequência. Nas avaliações serão utilizadas a LPS \textit{a Product Line of Model-Based Testing Tools} (PLETS) e M-SPLearning..	
\end{itemize}
\section{Execução}
\section{Análise e Interpretação}
\section{Disseminação}
\section{Ameaças à Validade}
\section{Melhorias Identificadas}
\section{Considerações Finais}


%\section{Cronograma}
%\label{sec:repres_e_fobj}
%As etapas que compõem o cronograma de execução desta proposta de dissertação são:

%\begin{itemize}
%	\item Etapa 1: Aprofundamento do referencial teórico sobre LPS, TBM e \textit{Smarty};
%	\item Etapa 2: Realização de uma revisão sistemática sobre TBM em LPS;
%	\item Etapa 3: Análise dos trabalhos relacionados;
%	\item Etapa 4: Escrita do Projeto de dissertação do Mestrado;
%	\item Etapa 5: Defesa do Projeto de dissertação do Mestrado;
%	\item Etapa 6: Especificação da abordagem \textit{SMartyTesting};
%	\item Etapa 7: Avaliação empírica da abordagem proposta;
%	\item Etapa 8: Divulgar os resultados por meio de artigos qualificados para eventos e periódicos;
%	\item Etapa 9: Escrever dissertação; e
%	\item Etapa 10: Defender dissertação.
%\end{itemize}
