\clearpage
\thispagestyle{empty}

\begin{center}	
	{\large AGRADECIMENTOS}
\end{center}
%\vspace*{0.5cm}
\normalsize
\noindent


Agradeço a Deus por ter me abençoado e dado forças nesta caminhada, assim como as pessoas que contribuíram de alguma forma na realização deste trabalho. Em especial: À minha esposa e companheira de jornada Cristiane, a qual me apoiou quando eu precisava, incentivando-me a cada desânimo. Minha mãe Ivone que sempre acreditou no meu crescimento pessoal e profissional, com palavras de incentivo e oração.

À minha família pelo carinho, apoio e incentivo.

Ao meu orientador professor Dr. Edson por todo o apoio, paciência com as minhas limitações. Sempre disponível quando precisei, agradeço imensamente pelos comentários, contribuições e sugestões para que este trabalho viesse a se concluir, serei eternamente grato pelos seus ensinamentos.

Gostaria de deixar aqui registrado minha estima e consideração pelos professores da minha graduação e especialização, representados pelo Dr Luiz Fernando. Também aos meus colegas de trabalho do Serviço Nacional de Aprendizagem Industrial (SENAI) aos quais sempre estiveram dispostos a contribuir de alguma forma para que este trabalho fosse possível.

Aos demais professores do curso de Mestrado em Ciência da Computação do Departamento de Informática (DIN) da UEM, agradeço pelos ensinamentos e formação de qualidade que recebi, todos foram fundamentais em minha jornada. Aos colegas das 3 turmas do PCC pela qual passei, onde pude receber um pouco do conhecimento de cada, todos foram fundamentais. Em especial: Viviane, André, Henrique, Eduardo, Renan, Helal, Luciano, Mariane e muitos outros que estimo de coração pela amizade e carinho. Não poderia esquecer o Dr Marcolino pelas contribuições iniciais ao meu trabalho, Dra Aline e o grande Dr Leandro ambos da Pontifícia Universidade Católica do Rio Grande do Sul (PUCRS) por ter me aturado nos últimos meses de pesquisa e que muito contribuiu para este trabalho.

Agradeço à Inês, por sua simplicidade, paciência, amizade, otimismo, alegria e disponibilidade em sempre poder ajudar de alguma forma.

Agradeço também à empresa \textit{No Magic} pela disponibilização da licença comercial para a ferramenta \textit{Cameo Enterprise Architecture} para o desenvolvimento de parte deste trabalho.

%E,por fim, à Coordenação de Aperfeiçoamento de Pessoal de Nível Superior (CAPES) pelo apoio financeiro a mim concedido.
