\chapter{Conclusão}
\label{sec:conclusao}
\pagestyle{plain}

\section{Contribuições}
contribuições desta pesquisa
\section{Limitações}
limitantes encontrados durante a pesquisa
\section{Trabalhos Futuros}

o que pode ser feito e qual o caminho


%Com o crescimento da prática de reuso de software e inovação nos processos de desenvolvimento, visando maior produtividade e qualidade, é necessária uma abordagem de teste que seja condizente com a utilização de reuso e principalmente relacionado as características particulares do domínio de negócio (variabilidade) de cada produto em uma LPS.

%Por isso, este trabalho se propôs a encontrar evidências que possam viabilizar a utilização de TBM em conjunto com a abordagem \textit{SMarty}
%Sendo assim, identificou-se a necessidade de uma ampla pesquisa sobre teste, e ao qual,TBM se tornou uma das práticas que podem ser trabalhadas em conjunto com LPS e \textit{SMarty}, Estudos iniciais indicam que existe possibilidade que este trabalho possa apontar um caminho para uma nova abordagem agregada na utilização de teste de uma LPS com mais eficiência e com foco em variabilidade do produto, algo que hoje é considerado por alguns estudiosos o maior desafio e um gargalo no desenvolvimento de produtos de uma LPS.
